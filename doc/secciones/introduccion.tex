\documentclass[../main.tex]{subfiles}
\begin{document}
\section{Introducción}\label{sec:introduccion}

El presente trabajo se analizan distintos metodos criptofráficos
para ciertas tareas como \textbf{cifrado}, \textbf{descifrado},
\textbf{hash}, \textbf{firma} y \textbf{verificación}. Estos
métodos tienen distintos algoritmos que cumplen la misma función,
sin embargo cada una de ellas tienen sus ventajas y desventajas la
cual es el objetivo de este escrito.

Los algoritmos ya clasificados que utilizaremos son los siguientes:

\subsection{Estándar de cifrado avanzado (AES por sus siglas en ingles)}\label{sec:estandar-de-cifrado-1}
Son algoritmos matemáticos o cifrados utilizados para ocultar
la información de manera que no pueda ser leída por los usuarios de
computadoras no autorizadas cuando se almacena o en tránsito.
Advanced Encryption Standard es un algoritmo simétrico, lo que significa
que utiliza una sola clave para cifrar y descifrar los mensajes. Una
persona debe tener en cuenta que una clave es simplemente una variable
inserta en el algoritmo selecciona aleatoriamente a los datos. Desde
AES se basa en una sola clave para hacer ambas tareas, es imperativo
que la clave sigue siendo secreto. Si un usuario no autorizado fue
capaz de obtener la clave, él sería capaz de leer todos los mensajes cifrados.
\begin{itemize}
  \item AES-EBC 256 bits\\
        ECB (Electronic Codebook) es esencialmente la primera generación del AES.\@
        Es la forma más básica de cifrado de bloques.
  \item AES-CBC 256 bits\\
        CBC (Cipher Blocker Chaining) es una forma avanzada de cifrado de cifrado de bloques.
        Con el cifrado en modo CBC, cada bloque de texto cifrado depende de todos los bloques
        de texto plano procesados hasta ese momento. Esto agrega un nivel adicional de
        complejidad a los datos cifrados.

  \item RSA-OAEP 1024 bits\\
        RSAES-OAEP es un esquema de cifrado de clave pública que combina el algoritmo RSA
        con el método Optimal Asymmetric Encryption Padding (OAEP)
\end{itemize}

\subsection{Algoritmo hash seguro (SHA)}\label{sec:algor-hash-seguro-1}

\begin{itemize}
  \item SHA-2 384 bits
  \item SHA-2 512 bits
  \item SHA-3 384 bits
  \item SHA-3 512 bits
\end{itemize}
SHA-2 es un conjunto de funciones de hash criptográficas diseñadas por
la NSA\footnote{\textbf{A}gencia de \textbf{S}eguridad \textbf{N}acional de los Estados Unido} y
publicadas por primera vez en 2001. Se construyen utilizando la construcción Merkle-Damgård, de
una función de compresión unidireccional construida en sí misma utilizando la estructura
Davies-Meyer a partir de un cifrado de bloques especializado.

SHA-3 es el miembro más reciente de la familia de estándares Secure
Hash Algorithm, publicado por NIST el 5 de agosto de 2015. Aunque forma parte de la
misma serie de estándares, SHA-3 es internamente diferente de la estructura similar a MD5 de
SHA-1 y SHA-2.  Es un algoritmo hash completamente nuevo que no tiene nada que ver con SHA-1 y SHA-2.

SHA-3 es una familia de funciones hash criptográficas especificadas en los Estándares federales
de procesamiento de información (FIPS) 202. Son hashes de permutación basados en el algoritmo KECCAK.\@

Del documento estándar:\\
La familia SHA-3 consta de cuatro funciones hash criptográficas, llamadas SHA3--224, SHA3--256,
SHA3--384 y SHA3--512, y dos funciones de salida extensible (XOF), llamadas SHAKE128 y SHAKE256.
En términos de alto nivel, las funciones hash SHA-3 transforman una cadena de entrada en una matriz
de estado multidimensional, luego realizan una serie de rondas de permutaciones en esa matriz de
estado que consisten en la aplicación secuencial de 5 funciones de permutación diferentes y luego
extraen una cadena de salida de longitud de esa matriz de estado. Los detalles de cómo realizar las
conversiones, cuántas rondas hacer y cuáles son las funciones de permutación se explican en el documento estándar.

\subsection{Firma y verificacion}\label{sec:firma-y-verificacion-2}
\begin{itemize}
  \item RSA-PSS 1024 bits\\
        El esquema de firma probabilística (PSS) o procedimiento de firma probabilística es un procedimiento
        de relleno criptográfico desarrollado por Mihir Bellare y Phillip Rogaway. En el modelo de oráculo
        aleatorio, se puede construir un proceso de firma seguro y verificable con el PSS a partir de una
        permutación de trampilla. En 1996, Bellare y Rogaway describieron la combinación de PSS con RSA
        como una permutación de trampilla en su artículo. En el modelo de oráculo aleatorio, RSA-PSS es
        existencialmente imposible de falsificar bajo ataques de mensajes elegidos (EUF-CMA)
        bajo la suposición de RSA.\@
  \item DSA 1024 bits\\
        El algoritmo de firma digital (DSA) es un estándar del gobierno de Estados Unidos para firmas digitales.
        El DSA se basa en el logaritmo discreto en campos finitos. Se basa en el proceso de firma de Elgamal y está
        relacionado con la firma de Schnorr. La transferencia del DSA a curvas elípticas se denomina ECDSA
        (algoritmo de firma digital de curva elíptica) y está estandarizado en ANSI X9.62.
  \item ECDSA Prime Field 521 bits
  \item ECDSA Binary Field 571 bits
        Es una modificación del algoritmo DSA que emplea operaciones sobre puntos de curvas elípticas en lugar
        de las exponenciaciones que usa DSA La principal ventaja de este esquema es que requiere números de
        tamaños menores para brindar la misma seguridad que DSA o RSA.\@
\end{itemize}




\end{document}
